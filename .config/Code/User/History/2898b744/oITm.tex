\documentclass{tufte-handout}

%\geometry{showframe}% for debugging purposes -- displays the margins
\usepackage[en-GB,showdow]{datetime2}
\usepackage{amsmath}
\usepackage{tikz}
\usepackage{soul}
\usepackage{color}
\DeclareRobustCommand{\hlcyan}[1]{{\sethlcolor{cyan}\hl{#1}}}

% Set up the images/graphics package
\usepackage{graphicx}
\setkeys{Gin}{width=\linewidth,totalheight=\textheight,keepaspectratio}
\graphicspath{{graphics/}}

\title{\Huge{Journal}}
\author{2023}
\date{}  % if the \date{} command is left out, the current date will be used

% The following package makes prettier tables.  We're all about the bling!
\usepackage{booktabs}


% The units package provides nice, non-stacked fractions and better spacing
% for units.
\usepackage{units}

% The fancyvrb package lets us customize the formatting of verbatim
% environments.  We use a slightly smaller font.
\usepackage{fancyvrb}
\fvset{fontsize=\normalsize}

% Small sections of multiple columns
\usepackage{multicol}

% Provides paragraphs of dummy text
\usepackage{lipsum}
\usepackage{xstring}
% These commands are used to pretty-print LaTeX commands
\newcommand{\doccmd}[1]{\texttt{\textbackslash#1}}% command name -- adds backslash automatically
\newcommand{\docopt}[1]{\ensuremath{\langle}\textrm{\textit{#1}}\ensuremath{\rangle}}% optional command argument
\newcommand{\docarg}[1]{\textrm{\textit{#1}}}% (required) command argument
\newenvironment{docspec}{\begin{quote}\noindent}{\end{quote}}% command specification environment
\newcommand{\docenv}[1]{\textsf{#1}}% environment name
\newcommand{\docpkg}[1]{\texttt{#1}}% package name
\newcommand{\doccls}[1]{\texttt{#1}}% document class name
\newcommand{\docclsopt}[1]{\texttt{#1}}% document class option name

\definecolor{halfgray}{gray}{0.55}

\newenvironment{loggentry}[2]% date, heading
{\noindent\textbf{#2}\marginnote{#1}\par}{\vspace{0.5cm}}

\def\?#1{}

\pgfmathtruncatemacro{\StartYear}{2023}
\pgfmathtruncatemacro{\EndYear}{2024}

\newcommand{\writetitle}{0}
\newcommand{\mytitle}[1]
{   \ifthenelse{\writetitle=1}{#1}{}
}

\newread\mysource

\newcommand{\monthToInt}[1]{%
  \ifnum\pdfstrcmp{#1}{January}=0 1\else
  \ifnum\pdfstrcmp{#1}{February}=0 2\else
  \ifnum\pdfstrcmp{#1}{March}=0 3\else
  \ifnum\pdfstrcmp{#1}{April}=0 4\else
  \ifnum\pdfstrcmp{#1}{May}=0 5\else
  \ifnum\pdfstrcmp{#1}{June}=0 6\else
  \ifnum\pdfstrcmp{#1}{July}=0 7\else
  \ifnum\pdfstrcmp{#1}{August}=0 8\else
  \ifnum\pdfstrcmp{#1}{September}=0 9\else
  \ifnum\pdfstrcmp{#1}{October}=0 10\else
  \ifnum\pdfstrcmp{#1}{November}=0 11\else
  \ifnum\pdfstrcmp{#1}{December}=0 12\else
  1 % Invalid month
  \fi\fi\fi\fi\fi\fi\fi\fi\fi\fi\fi
  \fi
}

\newcommand{\monthsections}{}
\newcommand{\createSectionIfNotExists}[1]{%
  \ifcsname monthsection-#1\endcsname
    % Section already exists, do nothing
  \else
    \section{#1}
    \expandafter\global\expandafter\let\csname monthsection-#1\endcsname\relax
  \fi
}
\usepackage{color}
\usepackage{soul}


\usepackage{todonotes}
\definecolor{clightgray}{gray}{0.95}
\presetkeys{todonotes}{backgroundcolor=clightgray, size=\small, linecolor=clightgray, bordercolor=white}{}
\let\annotate\todo
\let\annotations\listoftodos

\begin{document}
\pagecolor[RGB]{237, 232, 221}
\begin{minipage}{1pt}
\begin{tikzpicture}[remember picture,overlay,shorten >= -10pt]

\coordinate (aux1) at ([yshift=-15pt]current page.north east);
\coordinate (aux2) at ([yshift=-410pt]current page.north east);
\coordinate (aux3) at ([xshift=-4.5cm]current page.north east);
\coordinate (aux4) at ([yshift=-150pt]current page.north east);

\begin{scope}[halfgray!40,line width=12pt,rounded corners=12pt]
\draw
  (aux1) -- coordinate (a)
  ++(225:5) --
  ++(-45:5.1) coordinate (b);
\draw[shorten <= -10pt]
  (aux3) --
  (a) --
  (aux1);
\draw[opacity=0.6,halfgray,shorten <= -10pt]
  (b) --
  ++(225:2.2) --
  ++(-45:2.2);
\end{scope}
\draw[halfgray,line width=8pt,rounded corners=8pt,shorten <= -10pt]
  (aux4) --
  ++(225:0.8) --
  ++(-45:0.8);
\begin{scope}[halfgray!70,line width=6pt,rounded corners=8pt]
\draw[shorten <= -10pt]
  (aux2) --
  ++(225:3) coordinate[pos=0.45] (c) --
  ++(-45:3.1);
\draw
  (aux2) --
  (c) --
  ++(135:2.5) --
  ++(45:2.5) --
  ++(-45:2.5) coordinate[pos=0.3] (d);   
\draw 
  (d) -- +(45:1);
\end{scope}
\end{tikzpicture}
\end{minipage}
\hfill
\begin{minipage}{.99\textwidth}

\maketitle
\end{minipage}
\clearpage
\include{books}
\clearpage
\foreach \Year in {\StartYear,...,\EndYear}
{   \foreach \Month in {January,February,March,April,May,June,July,August,September,October,November,December}
    {   \foreach \Day in {1,...,31}
        {   \IfFileExists{\Year/\Month/\Day}
                {   \createSectionIfNotExists{\Month}
                \openin\mysource=\Year/\Month/\Day.tex
                    \read\mysource to \firstline
                    \closein\mysource
                    \xdef\writetitle{1}
                    \DTMsavedate{mydate}{\Year-\monthToInt{\Month}-\Day}
                    \begin{loggentry}{
                    \DTMusedate{mydate}
                    }
                    {\firstline}    
                        \xdef\writetitle{0}
                        \input{\Year/\Month/\Day}
                        \xdef\monthh{\Month}
                        \lhead{\monthh}
                    \end{loggentry} 
        }
        {   % files do not exist
        }

        }  
    }
} 
\clearpage
\bibliography{references}
\bibliographystyle{plainnat}
\end{document}
