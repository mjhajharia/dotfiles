\noindent \mytitle{still getting the hang of this thing}

So from the time I last wrote, I almost had a breakup with Alexey, we had a serious conversation, he made me laugh a lot. I drank a lot of gin, Aman fucked up with Aarushi, I don't know what to make of him. The selfish part of me just wants to cut him out of my life because I don't care for the drama. Dhruv sent a rather sappy email. Work life is a complete ruin, so I come here to seek some semblance of sanity.

Now, before I examine and resume the stuff from last day's entry, here's some new stuff: I have a theory, that there's something that I've subconsciously found rather appealing in aristocratic people, or "old money" as tiktok would call it. Thinking of all this from Caroline Blackwood's biography, titled "Dangerous Muse", anyway, it's not so much as the beauty, charms or riches that they possess, but the ugliness of some plebian worries that they are not tarnished by. So my theory is, if I even mentally acted like I was someone raised with utmost elitism, had nothing to be embarrassed about in those senses, have enough of a fortune to not worry about work: then I could truly do something extraordinary, in fact maybe that's the only way I could do something extraordinary. Because I'm too fixated on my deep sense of shame that comes with the clash of reality where plebian concerns exist, and my ideal self not being a person who is formed by those dents. So, as far as I can tell, the only way to stop feeling suffocated, is giving in to the delusion, fake it till you make it or something. 

So I should now soon think about the watered down effects of this, one thing that has happened since I met Alexey is, for the first time, whether in writing or in thoughts, I've developed a strange ability to take myself seriously, where I'm acutely aware of my own voice, as if it's this thing that is only possible when you're deeply seen by someone else. I'm really in love with him. 

Anyway, so I don't quite know what to make of the whole theory of delusional thinking that I wish to practice. I am also beginning to need a drink almost everyday, I am really not having it with everyday life, I think I need to reach a point where I want to quite seriously throw myself into something. I do feel like writing is almost my calling and it's very hard to do everything the way I want to. It's hard to go back to anything once you find the ecstasy of literature. So, I have to choose: here's the thing if I could either do math or read, and doing Math meant no more reading ever, I think I would quit Mathematics. But, I wouldn't be happy, or I don't know maybe that's what I want to believe. I don't quite know. There's this stability in Math, and then literature is this mistress I have because I am a cheating, manic Robert Lowell, and Math is Elizabeth Hardwick and literature is of course, Lady Caroline. I have got to choose. This is a false dichotomy, right before I met Alexey this didn't exist, I still wasn't showing up. I just don't show up, even for literature, I am just not doing the thing with grit, I think that is the one thing that I need more than anything else at this point in my life. I have spent my whole life shying away from uncomfortable things. I stayed in relationships because breakups were too much for me, I shy away from this sense of conflict and difficulty, so no wonder I'm wounding up a mediocrity, I need to take risks, for me that means hours of plunging myself into things. I have got to do this for myself. Okay, Grit it is.

So now, we've covered why I'm not doing the things I need to, as well as what I can do to undo that. As far as distractions and priorities are concerned, lists and organization are just anpother means to procrastinate because they have this sense of familiarity where I know where I'm going, research doesn't. Guess what, grit means getting used to that discomfort. This was good, I'll drink, I'll do whatever, I don't need too many rules, I can have chaos, I don't need a narrative, I just need grit, to be a force of some kind. That's the one thing nothing cool happens without. Think of Paul Graham's essay on people being Lazy and careless with things that don't utterly matter to them, if I plunge myself into hardwork slowly I will figure out what matters to me, as long as I make active choices. I don't need to create a narrative, narrative follows life, not vice versa, that's all.

Alright, signing off here, going to Math Journal now, although here's a reminder to myself\annotate{write about The Dangerous Muse and excerpts there when I get a chance}. Let's do this. Graduate applications are super close. I need to have a list, and email people. There are three possible groups for applications - Applied Math, CS Theory, Probability and Statistics. \annotate{2-4pm everyday: work on statements, choosing schools and emailing people} Now, before anything and everything I need to focus on two things: Math GRE and Transforms Paper.