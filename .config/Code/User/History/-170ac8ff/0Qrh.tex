\subsection{Whitening Transforms Notes}

\begin{itemize}

\item Covariance Matrix: $\Sigma$
\item Eigendecomposition: $\Sigma = U\Lambda U^{H}$, where $U$,  $\Lambda$ consist of eigenvectors and eigenvalues of $\Sigma$.
\item Cholesky Decomposition: $\Sigma^{-1} = LL^{H}$, where $^{H}$ is the conjugate transpose or adjoint, for real matrices this is just $^{T}$
\item Also, $\Sigma^{-1} = U \Lambda^{-1} U^{-1}$
\item Now the unique inverse matrix square root of $\Sigma$ is defined as  $\Sigma^{-\frac{1}{2}} = U\Lambda^{-\frac{1}{2}}U^{T}$
\item We assume that the eigenvalues are sorted in order from largest to smallest value
\item For a whitening matrix $W$, the Transformed Hessian is constructed as $H_{T} = W^{T}HW$
\item $W_{ZCA} = \Sigma^{-\frac{1}{2}}$, can also be rewritten as $(\Sigma^{\frac{1}{2}} L) L^{T}$ (QR Decomposition)
\item $W_{PCA} = \Lambda^{-\frac{1}{2}}U_{T}$
\item $W_{Cholesky} = L^{T}$
\item SVD: $A=U \Sigma V^T$, where $U$ and $V$ are unitary and $\Sigma$ is a diagonal matrix with non-negative diagonal entries and $v_i= \pm u_i$
\item Eigenvalue/Spectral(Normal or Real Symmetric Matrix) decomposition: $A=X \Lambda X^{-1}$.
\item Orthogonal decomposition: $A=P D P^T$, where $P$ is a unitary matrix and $D$ is a diagonal matrix.
\item Schur decomposition: $A=Q S Q^T$, where $Q$ is a unitary matrix and $S$ is an upper triangular matrix.
\item $U, V, P, Q$ are unitary matrices | orthonormal when $A$ is symmetric
\item $\Sigma$ is a diagonal matrix with non-negative entries, which consists of singular values $|\Lambda|$ 
\item $\Lambda$ is the diagonal matrix of Eigenvalues
\item $X$ is the orthogonal matrix of Eigenvectors
\item $D$ is a diagonal matrix
\item $S$ is an upper triangular matrix | diagonal matrix for symmetric $A$
\item When $A$ is symmetric, then $S$ is a diagonal matrix and the eigenvectors can be chosen to be orthonormal, $U=X$ and $\sigma_i=\left|\lambda_i\right|$.
\item Singular Value, Eigenvalue, Orthogonal and Schur Decomposition are equivalent for Symmetric $A$, if eigenvectors are defined only up to a sign
\item If eigenvalues are negative, then naturally multiply by a factor  $-1$ to get the singular values
\end{itemize}