\noindent \mytitle{Essay - A testament to the tiffies of taste(or lack thereof)}

Insofar as taste is concerned, for a while I have discerned it as something beyond arbitrary negation of notions to seek out specific narratives to live your life along. It reeks of history and effect far beyond that. The other day, in the most cliched way I was thinking of this with regards to politico-personal opinions. This was triggered by reading \href{https://en.wikisource.org/wiki/Studies_in_Pessimism/On_Women}{On, Women by Arthur Schopenhauer}. Nothing surprising there, history is strewn with the wreckage of those who have minded beyond reason the lives of others, to paraphrase Woolf. For some reason, for the first time, I seriously decided to face how ridiculous it is, that we have no real explanation for how this can be reconciled with relatively qualified intellectual work that many people like him have produced. This is not to say good art can be a motif for good morals, that is reductionist of course, but also I'm not even entering the trap of good morals here. My essential trigger from the essay were the serious logical fallacies, I could still in some stunted way appreciate a set of deductions I didn't agree with or didn't wish to agree with, branching out from an assumption that was not based on truth. Schopenhauer doesn't give at all here, none of it, there's just a bunch of hodgepodged theories(too dignified) about his opinions, which is fine, the ick is when it's considered scholarly work. I would imagine every accomplished public intellectual with certain neuroses would hold themselves to a higher standard than that. 

Now, that leads us to a bit of digging, of course there are the banalities of personal experience, the essay above(if you get through the painful act of getting through it somehow, as if you're watching ozymandias having a concussion on paper) seeks inspiration from his relationships from important women in his life, nothing special there, reads akin a 4chan post. I am simply trying to trace the point of diversion where impressionable teenagers decide to go down the punk/rebellious/somewhat-liberal/twitter-raging-activists/hardcore-leftist/art-kid route vs the english-public-school/snotty-merit-is-all-that-matters-as-a-cover-for-their-beliefs/regular-4th-amendment-people-on-social-media/rationalists-ea-tech-twitter/andrew-tate-esque cults, yes of course they're cults. Some personal disclosure here, although I like to sit with nabokov on the back bench on these issues(
“There is nothing in the world that I loathe more than group activity, that communal bath where the hairy and slippery mix in a multiplication of mediocrity.”), for all practical purposes I could binned a leftist. In reality, I'm just the impostor child of Moliere's Misanthrope. 

About the divide, I think it's a matter of taste, rather than opinions or affinities. Think of it, there's a certain 16 year old reading Nietzsche and due to a common human weakness of doing anything to hold tight to the hard earned and rare found acceptance, you see that person deciding to mirror their taste, it's not their opinions they mirror. This not-so-subtle linguistic distinction is quite important, you aren't exactly trying to mirror opinions, the 16 year old is also in possession of god's given ego( of course they hold the pump for inflation in their own tiny hands). A certain taste being the root of these divergences is the only way to possibly explain the contradictions, it's too reductionist to say "oh they're priviliged and bad people", I've seen plenty of priviliged priviliged-leftist-art-kids to know this is not the real reason (of course, I can't unsee the correlation, it's like having common ancestors, not really causation or correlation, just taking the statisticians on a roller coaster). By contradictions I mean this -- as far as I can tell, almost all the people in the latter end of the dichotomy(a vague word for this is the socially conservative, although there can be 5-6 framings for this dichotomy at the very leaast, point is when superimposed there's a massive overlap and you almost get a thick line, the really-trying-hard-to-be-interesting ones obviously lie in one of the faint lines in the superimposed venn diagram figure.)

Of course when I say taste, this has correlations with the notion of considering opinions or something else as the proper defining reason, such as socioeconomic class. But there exist enough contradictions, to think otherwise, and I for one would like to move away from thinking about these things in terms of "people are stupid and get swayed", even if that were true, it's worthwhile to think about the process of swaying people. In some way it's an admixture of accepting that there's a massive brain-frying-capitalism effect on society, lack of media literacy etc(no more than the 20th century mind you, people listened to nazis on radio, all bright eyed bushy tailed). 

The contradictions I speak of with admirable glee here look something like this - going back to the essay On Women, it's a good example that serves a testament to my observation, which is the punchline of this essay - as opposed to what would be the natural conclusion of the weird and fake dichotomy(I think it's good to assume fake dichotomies as true, in an ASFOC manner, to latter establish contradicitions and dismantle them) of the conventional(controversial) feminine vs masculine or left v right and so on lines of thinking, one would usually go about justifying them as follows: the former cares more about empathy and less about logical deductions and is therefore morally sound and the latter is a crude devil's advocate sort of thing that only care about rational thought processes. I beg to differ your honor. For a quick nonsensical thought experiment, think of incels and their posts, irregardless of the merit or morality of the content, they ensure they're playing a negative sum game where they stray further from their goal. Now think of a conventional girl, the kind hated by not-other-girls, attractive, feminine, (subjectively) boring and so on: they are very clear on what they want, and do exactly what's needed to get it. They can be petty, but they're so so far away from irrational or stupid. This is also related to the popular-on-the-web narrative of the selfish-women-who-win. This can be diverged into many other areas of why men are actually more hysterically irrational based on emotional premises, but that's an otter to decipher for another day, let's let it swim for now, shall we. 

So now, we're left with no defenses about inherent tendencies once we cast aside explanations other than taste about these phenomenons, additionally, now I'm forced to explain the structure I've spent all this time adorning (a rather tedious task). Let's start with how when I was 17 this notion of equity which was valued by all my internet friends became a semiotic identifier of truth for me, also going back to the thing about the dichotomy of thinking about logical deductions, this very well explains the moments where you see people on leftist twitter saying absolutely ridiculous things for the sake of virtue signalling (e.g. therapy speak), it's mostly a consequences of thinking a bit too much about the effect of pragmatic deductions from their statements. The problem with this approach is, it considers other people as radiation-absorbing-mannequins that take in light from the shrine of wisdom their words throw out, right? Why else would you fail to see the self-absorbed way of virtue signalling that you're performing(the worst way btw). Anyway, point being, wouldn't you expect the so called rational souls to be better at this. Maybe this means everyone is a rationalist at heart **Rationalist crowd cheers**, thinking of Julie Delpy going "everything we do in life is just a way to be loved a little more", now people take this in two ways - preserving the stuff inside, or accumulating more at the cost of losing the stuff inside at an accelerated pace. The conservative purists, living in the castle of camelot, find it in preservation by dodging certain digs at the core of taste through any means necessary. While others, othered by Arthur find it in raging against, thinking of the peasants othered, both testaments to the tiffy of taste.

Ofcourse, this is incomplete, selfishly so, I need to come back here again. Of course, anon is a woman. Of course, you are allowed to sneer.
\clearpage