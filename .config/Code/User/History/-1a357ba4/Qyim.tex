\noindent \mytitle{Beginning}
-
\noindent I hope to make this Journal a memoir of sorts, I'm afraid I'm forgetting myself and my life as a whole. This is just an attempt at self-preservation. God help me, because it's not easy. Let me take this one step at a time, I am overwhelmed, there's a whole bunch I wish to put down into words, I'm going to afford myself some chaos this first time, I promise to be more coherent from now on. The beginning is always exhilarating, no fuck this, I am not going to dip my toes and check the territory, I need to dive in headfirst. Here are some things I wish to do: 

Firstly, Write down whatever I can remember of myself from before this, I need to compartmentalize past to be alive. This journal will be nothing but a memoir of the present. I like to keep these people separate, so there is something left that can come and knock at my door at 4am\footnote{On keeping a Notebook, from Slouching Towards Bethlehem by Joan Didion}. So that will not be a part of this particular Journal document, so this item in the listicle is essentially an exercise in circumscribing negation.

Secondly, Get a fucking grip. I have been so terrifyingly passive with myself. I need to voraciously pick myself up and create the rest of me from the silhouettes that I see in nightmares and hallucinatory visions, soon I'll lose the silhouettes too. Betrayal lingers around all the edges of memories.

So now, another vessel of procrastination is filled, where I talk about what I'm going to do within the thing I'm doing. It's time now. I feel lost, I feel terribly lost, it's horrifying as if I've been thrown into a vortex, I've sculpted out so many selves for so many other people and circumstances that I don't have the faintest ear left to hear the muffled voice inside me. I am hoping to plan an emergency mission now, to save that kid in it's last moments. It's a matter of utmost urgency, unfortunately that child can't be made undead, and once it's dead, all these grown selves I've made like voodoo dolls, they will be left without a witch or puppetmaster. They will have to deal with themselves, and I don't hope to withness that bloodshed, because if there's something in common with all of them, it's the vicious self-assertion. One of my fundamental problems seems to be this urge to create narratives out of my own life, as if I'm already someone reading my memoir a few decades later trying to decipher the life I lived, life is meant to be lived first, recorded later. I'm thinking of the time my therapist(Snehal, blocked now) told me "Your life is not a movie, you aren't a character there, nobody is watching, nobody is constantly thinking whether it makes a good story" (this paraphrasing is somewhat exaggerated with artistic liberties of a good translator). It's as if I'm infected with the gender neutral version of the Margaret Atwood excerpt.\footnote{ “Male fantasies, male fantasies, is everything run by male fantasies? Up on a pedestal or down on your knees, it's all a male fantasy: that you're strong enough to take what they dish out, or else too weak to do anything about it. Even pretending you aren't catering to male fantasies is a male fantasy: pretending you're unseen, pretending you have a life of your own, that you can wash your feet and comb your hair unconscious of the ever-present watcher peering through the keyhole, peering through the keyhole in your own head, if nowhere else. You are a woman with a man inside watching a woman. You are your own voyeur.” -  Margaret Atwood, The Robber Bride } So here's a bunch of concrete stuff to work on:
\begin{itemize}
    \item Get out of My-life-is-a-movie syndrome.
    \item Figure out your real vocation.
    \item Figure out why you can't seem to make progress on that vocation.
    \item Figure out how to make that progress.
    \item Make it happen.
    \item Figure out what you want from other people, what you really want, and whether or not(and why) you want certain people around.
\end{itemize}

I'm going to start with the last item in the list, because it seems a bit separate from the rest of the list, except the first one, which is actually the seed all of these problems germinate from. I just heard this line\footnote{
    \includegraphics[width=0.2\textwidth]{static/masakali.png}
} from the song Masakali and I feel alive, this is it. This is it. I need to stop seeking some sort of a persona of seriousness. So maybe afterall I just needed some kind of an emotional kick to get out of the need to create narratives, I associate my overbearing obsession with femme fatales with the need for narratives and for seeking out pain. I don't need that. Yes, there is a certain side of darkness that I can't run away from and yes it carries intrigue, and yes I will fall into that puddle every once in a while and it's better to embrace it in those moments than pretending to be clean, but it's also rather stupid to daydream about those puddles and devising ways to fall into them. Alright, I guess this was the growing up I needed to do, this is why I had the conflict with my blue and black hair. Now coming back to Alexey, I do love him, or a statement with greater certainty is - I was stupidly in love with him in the that summer in New York. It is without a doubt the coolest thing my life has witnessed, I am in awe, I can't believe I got to live that. Now, there's things about him that bother me, the issue is I don't know if they are things I notice to find rational reasons to throw him out of my life now that I've concretely noticed the loss from the summer and I am a maniac who wants to keep living that summer on repeat. He is again someone very different from me politically, he's not a complete asshole, but he's not too good either. The problem is, despite somewhat knowing this in the back of my mind over the summer, I embraced the Bonnie and Clyde fantasy and I had this vision of us against the world and so it didn't matter. The problem is, now he's decided to plunge into reality, and he's in so deep that he can't even see me floating in the sky. This is to say, I am not excluded from the reality glasses he has put on, he doesn't take them off for me, he has embraced his reality and it isn't us vs the world anymore, it's him and the world now, which includes me. This forces me to unwillingly get out of the zone I was in during the summer and put him back in reality as well, as this asymmetry bothers me. The moment I do that, this relationship is a clear failure, an appendage that's redundant. Hence the splitting, my head realizes this and keeps telling me to break up, then I tap into "the real me" or the emotional side that wants to keep holding on to the self I was during the summer, and that person doesn't care about any of this, of course, because it's us vs the world. Here comes the problem of narratives, I can't seem to make decisions on nuanced, complicated storylines. I need a narrative. This is a false dichotomy, as Alexey would love to say. Yes, there's the change in him, but whether it's reactionary or not the same change has also seemingly irreversibly happened in me, and it's not a linear trajectory so we don't quite know where the change will lead us. So yeah, it doesn't have to be this or that, it's this disillusionment of realizing how in my moral strictures he doesn't qualify to be a good person or someone I would be friends with, that I keep seeing the writing on the wall. Truth be told, I've changed too, do I really care so much about my friends being "good people" anymore, I've always cared more about being charmed and loved, and that's just how it is. But realizing he crosses boundaries 17 year old me cared a lot about, makes this feel like a transgression which requires an equal reaction. These are not good ways to solve problems.\annotate{Get out of narratives, be a person.} So what's next then, how do I actually deal with this, well for starters, I love him. He's not a bad person, I'm not so sure of my views even, I should not be swayed anymore, I should stop attacking him, I should provide him the same human dignity I give myself when I'm confused about my views\footnote{Think back to the time I was 15 and had that \#metoo twitter fight.}. So now, concretely on a day to day basis, I will not be a persona, but a person. I will do as I please, Masakali all the way. I will not be thinking about what makes for an impressive partner, rather focus my energy on being selfish, I haven't been selfish, instead \hl{I've been selfishly selfless, which is always a bad idea, selfishly selfless is the way to go. Which in reality means, being kind, caring about my own life, forgetting about personas, living really, and loving.} My job isn't forming the right judgement of my partner, and this need of obsessively getting it right, letting my control issues spiral, they're such a big part of me, I have got to let go. \underline{(Timestamp, 12:55pm:)} Okay I talked to him on meet, I mean nothing that I talked about just sort of existed, and it was cool, I feel alright, and I love him and I refuse to overthink this. If it's the thing for me, it'll work out just fine, else not. The premise of the talk we had about the December deadline for this relationship to work it's way out of the slumber means I keep putting in normal effort, refuse to take all the emotional load and lighten it for myself, thinking about it and letting my control issues gobble this whole thing up is the opposite of that and is quite counter-productive. So now we address the last bit of my control issues\footnote{Music, Literature, Art; Collecting, organizing, folders, files, images et al}: So here's the thing, I really do not have time right now, and while these are things worth doing, as I promised(and failed to keep it) myself a while ago, I will do these things but only when they feel joyful and immediately useful - such as this journal template. Now for posterity(other selves of meenal, of course) I shall make a listicle:
\begin{itemize}
    \item Organizing Books (librarything, goodreads, calibre, PDFs, Kindle, iBooks, Paperbacks, annotations).
    \item Music(Spotify, playlists, Apple Music, mp3 files, Vinyls/CDs)
    \item Misc. (Files, Folders, Photos, Documents, Notes, Calendar)
    \item Web things: Cleaner Web presence
    \item Sense of clutter: Decide on social media (immediate)
    \item Things worth doing immediately: Get the fair clustering PDF out of the web \annotate{(task)}
\end{itemize}

So now that we know that the last two are useful, the last one is simply a matter of doing, now about the penultimate item in the list: Pro/Con list time, time to tap in to the inner Rory Gilmore, yesterday I finished the series by afternoon on an all-nighter and it gave me some closure, I need a moment like when Jess came and told Rory she was being someone else and not who she was, maybe you need people who knew you when you were younger to tell you these things\footnote{I think Natalia Ginzburg or Joan Didion said something about this, can't quite remember, most likely somewhere in The Little Virtues.}. When I was younger, I had a slightly dark side but I cared about integrity, I cared about writing, I also found Math and it made me feel excited, and it gave me something very important - stability. It feels like literature(the dichotomy and duality of the self is back again) just throws me in deeper into the vortex and math provides anchors to hold on to. So yeah, I need to be a writer in the off hours to cater to some urges and selves I have, but only when it helps and when I need to, and care about quality. As far as a vocation goes, it's Math. So yeah, now, another thing, is there any particular thing in Math that I feel I deeply care about in an intuitive way? I think all I can gather up right now is there's a side of taste in problems etc that I like and dislike. Okay before I meander I should go back to the social media thing: so twitter has the reach stuff that's useful, erratumaddendum has the curating stuff that's useful, Instagram is garbage but I don't use it often and the real thing is letting go of my control issues and not having a persona so the account can stay I just won't be a frequent user, so that decision is made. Again the decision is just let things fucking be and get over your persona control issues. And the first three items in the list are non essential and can be iterative, I will cater to them when I have time and energy and interest, that's about it. Now back to the original listicle, i.e. concrete stuff to work on:
 



