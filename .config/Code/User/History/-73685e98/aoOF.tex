\section{Flatiron Transforms}
\label{sec:transforms}
\subsection{Simplex Paper}

% \textbf{quantities}: ess, rmse , leapfrog steps, hessian - cn, condition number, kl divergence, iters to 95 percent density, autocorrelation time.\\
% \textbf{plots}: ess-cdf, density; rmse plot; spectra plot; metric geometry plot\\
% \textbf{models}: Dirichlet Symmetric, Skewed Dirichlet (Ascending and Descending) and Log Normal Distribution.\\


Bob: transient bias (iterations to reach central interval of log density)
autocorrelation time at stationarity using NUTS
approximation quality of Laplace at mode in unconstrained space measured by projecting back to constrained space. This is where smaller is better, but we can also plot the inverse, which is ESS / iteration. I'd rather not frame purely as transforms because we have the suriective case where it's transform plus identifying distribution. So I suggest we use "parameterization" instead for a bit more generality. I think any theoretical properties of the parameterizations can go into the section on that parameterization following the definitions. I think we should reproduce (and of course site) the figure on isometry-that's super helpful.